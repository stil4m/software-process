``Whilst understanding fully the pressures that the project team were under to achieve a quick and successful implementation ...
knowing that there were so many potential imperfections in the system” \autocite[4002]{officialreport};
a statement that unfortunately summarizes the London Ambulance Service (LAS) case to quite a frightening degree of accuracy.
What architecture was used during this project and to what extend could things have been different if changes were made to the architecture?
In this document we report on the architectural analysis that has been done in regards to the LAS case.

The authors assume that the reader is familiar with the LAS case.
If this is not the case, then we advise reading our previous report
``Failure of the London Ambulance Service, In-depth analysis of the quality assurance and tender process for the London Ambulance Service incident and how modern methodology might have saved the project''.

The first chapter discusses the dominant decomposition used in the LAS case.
It also talks about how the dominant decomposition relates to the organization.
The following chapters analyze three major failures in the system from an architectural point of view.
The chapters show an analysis of where the applied architecture helped and what changes in architecture the project could have benefited from.
