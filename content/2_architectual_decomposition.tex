\section*{Architectural Decomposition}

\subsection*{Dominant System Decomposition}

The documentation we analysed makes no direct mention of the design methods and architectural patterns applied in the LAS case.
Therefore, in determining the architectural decomposition certain assumptions will be made based on the information that was provided by the documentation.

We believe that the dominant decomposition of the architecture was component-based.
The different components in the system (as shown in Figure \ref{fig:figure1}, appendix A) were created by different companies:
\begin{itemize}[noitemsep]
\item The Automated Vehicle Location System(AVLS) was created by Datatrak.
\item The Computer Aided Dispatch (CAD) software was created by Systems Options (SO).
\item The Radio Interface System(RIFS) was created by SOLO.
\item Network infrastructure was delivered by Apricot.
\end{itemize}

This does not directly provide evidence for a component based architecture, as members of different companies could have worked together in teams with a single responsibility
(for instance a service in a service-oriented architecture).
However, we have found no evidence of the existence of such teams, which is something we believe, would have been mentioned in the official inquiry report \autocite{officialreport} if it had existed.

We also considered expert based composition to be the dominant decomposition due to each company working on a part of the system of which they were experts.
However, as with a service-oriented architecture, one would expect to see teams with members from different companies in an expert based decomposition.
The CAD software for instance required expertise from all companies, as it had to interface with all the different systems.
If an expertise based composition had been used, multiple teams with a specific expertise would have worked on the CAD software.
As SO (which was an expert on Geographical Information Systems \autocite{techsum} but not an expert on integration projects of this size \autocite[3078]{officialreport}) was solely responsible for the CAD software,
we conclude that the expert based decomposition was not the dominant decomposition in the LAS project.
We thus find it reasonable to assume that a component-based architecture has been used.

Although we believe that at top level the dominant decomposition of the architecture was component based,
we have identified other decompositions that were dominant within specific components.
This is in line with the content of a paper written by Kramer and Wolf where they stated that
``Design method engineering assumes that a single design method is not suitable for complex systems,
but rather that component-specific design methods must be chosen''\autocite[26]{kramer1996succeedings}.
In the following paragraph we will look at the dominant decomposition of the architecture in the file server component.

\subsection*{File server}

The LAS architecture contained a central file server, where all the data from the different components of the system, such as the location of the ambulances, was stored.
Within this component, we identify the shared data pattern as the dominant decomposition.
When this architectural pattern is used, it should be taken into account that the storage may be a performance bottleneck and a single point of failure \autocite[chapter 13]{softArchitInPractice}.
The crash of the system on November 4th 1992 shows how the file server was a single point of failure in the LAS architecture as a memory leak in the software on the file system lead to a complete lockup of the system.