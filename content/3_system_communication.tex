\section*{System Communication}

As mentioned in the Architectural Decomposition, we concluded that the component-based architecture was dominant.
This aspect strongly returns in Figure \ref{fig:figure1} located in appendix A,
which shows the communication lines between the different entities in the system.

We believe the component-based decomposition brought some advantages, such as the possibility to replace components.
LAS decided to use hardware from the previously failed project \autocite[8]{kramer1996succeedings}.
We assume this requirement is one of the reasons a component-based architecture was used.
A component implementing a certain interface would be introduced to encapsulate the old hardware and corresponding software.
This gave LAS the possibility to easily replace the legacy hardware, as only the implementation of the interface would have had to change.
During the tender process the consortium decided to change the supplier for the AVLS \autocite[3073]{officialreport}.
Suppose that SO already started on the development of the CAD system, then the use of component interfaces reduced the amount of extra effort (changes flowing through the system).
It is important to note that this advantage only applies if the interfaces are well defined.
If implementation details are reflected in the interface one still needs to change the components that use this interface when switching to a new implementation.
Another limitation on interfaces is that programmers need to reconsider them when new functionality is introduced\footnote{For example, making use of new functionality provided by new or more advanced hardware.},
resulting in the loss of the purpose of having interfaces at the first place.

Another advantage of the component-based decomposition is the ability to replace certain components with stubs.
A problem was the readiness of the RIFS hardware \autocite[3074]{officialreport}.
The implementation of the integration with the RIFS was blocked, as the hardware was not completely finished.
As different components depended on this functionality it was a crucial part of the system.
In order to prevent this from blocking the work on other components SO started with the component interface and created a dummy implementation.
This allowed them to continue to work on the other components instead of just waiting until the hardware was finished.
We believe this also helped when LAS decided to go for a phased implementation of the system.
By replacing components that were not ready or had a low priority with stubs SO could focus on the most important components.

We suspect SO created a deployment view in which components and their relation were specified.
This components overview allowed developers of a specific component to identify how components were related to each other.
For example, if a change to a component interface was made one could quickly identify which other components required a change.
Additionally it could have been beneficial for the planning of the test process.
For example, to identify which components could be tested separately.

One of the problems of the LASCAD system was that the AVLS frequently did not transmit the (correct) location of ambulances.
This resulted in incorrect allocations by the system such as not sending the closest vehicle to an incident \autocite[4022]{officialreport}.
``The Datatrak system as it was installed was likely to be as reliable as any AVLS ever could be in an urban environment such as London at that time'' \autocite[3133]{officialreport}.
The knowledge that the AVLS’s did not always submit (correct) locations was part of Datatrak’s expertise.
SO which depended on the location data had no knowledge of this problem.
This was a consequence of the distribution of the work between the different companies.
The work was distributed in such a manner that the companies worked isolated from each other, resulting in a lack of alignment between SO and Datatrak.
SO’s dependency on Datatrak’s data should have indicated that knowledge sharing should have taken place prior to development on this component.
This lack of knowledge could have been identified with a (technical) use case diagram in which the AVLS would be modelled as one of the actors.
Assuming SO would have made this diagram, it would still not contain the alternative scenario in which no or inaccurate data would be provided.
However, if they used this diagram in their communication with Datatrak, Datatrak could have pointed them on the missing scenario.
This diagram should have been part of the process to facilitate communication between the involved parties.

When SO decomposed the architecture in different components a problem arose, which was not identified until the system ran in production.
There was no clear specification on the input validation that had to be done for each component.
As stated in the official report, ``There had been no attempt to foresee the effect of inaccurate or incomplete data available to the system (late status reporting/vehicle locations etc.)'' \autocite[4001]{officialreport}.
The logical view of the 4+1 model would have helped as this problem could have been identified with the use of sequence diagrams on a component level.
Using a sequence diagram one can trace the user input from the client through the different components.
The fact that no validation step was present should have indicated the lack of validation.